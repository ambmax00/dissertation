\chapter{Conclusion and Outlook}

The development of local excited methods is an active area of research. It combines two concepts, namely local electron correlation and electron excitation, which are already very demanding by themselves both from a theoretical and computational point of view. While the earliest attempts date back to the early 2000s, the last decade has seen a substantial increase in interest. Prior to this work, attempts to a local treatment of electron excitation were limited to local molecular orbital and (local) natural orbital approaches. The spin-opposite-scaled second-order algebraic diagrammatic construction method with Cholesky decomposed densities and density fitting (CDD-DF-SOS-ADC(2)) is the first local excited method beyond CIS which is based on an atomic orbital formulation. By exploiting the sparsity of both the AO ground state and transition densities, combined with modern local density fitting techniques, CDD-DF-SOS-ADC(2) drastically reduces the scaling compared to canonical SOS-ADC(2). Linear scaling is observed for the "ideal case" systems of linear carboxylic acids, and near quadratic scaling is obtained even for very electron-dense systems such as hydrated formamide. This work also presents the first linear scaling implementation of the quasi-robust density fitting scheme, which proved to be a crucial ingredient for decreasing the computational effort of CDD-DF-SOS-ADC(2) with similar accuracy to standard density fitting on the order of a few $\mu$Hartrees. 

The CDD-DF-SOS-ADC(2) method is implemented in \mchem{}, a quantum chemistry software specialized in algorithms exploiting the sparsity of the atomic orbital basis. Sparse matrix algebra and tensor contraction is offloaded to an external library called DBCSR, which plays a crucial role in the success of CDD-DF-SOS-ADC(2). The method is highly parallelizable and suitable for distributed memory systems.

While CDD-DF-SOS-ADC(2) theoretically scales linearly in the limit of infinite system size, its applicability is still hampered by a few obstacles.

First, the method is plagued by the same problems as most atomic orbital based methods: the computational effort increases significantly for large basis sets with diffuse basis functions, which are necessary to obtain accurate excitation energies. Diffuse functions severely impact the scaling of CDD-DF-SOS-ADC(2) by reducing the sparsity of the matrices and tensors, effectively negating the benefits of the method. These effects are further exacerbated by the complexity of the working equations compared to the ground state energy expressions. The low-scaling regime can often not be reached for electron-dense systems, and memory resources are quickly exhausted, even in high-performance computing environments. This is an inherent weakness of AO methods, and will need to be addressed in future iterations of CDD-DF-SOS-ADC(2).

Second, the diffuse basis functions are also less than ideal for local density fitting methods. Density fitting often involves some form of matrix inversion or matrix decomposition, which are both very sensitive to linear dependencies. Diffuse functions can often reduce accuracy and lead to convergence issues. 

Third, while the DBCSR matrix library is quite efficient, the tensor library built on top of it has a few drawbacks. Tensors need to regularly reordered which effectively doubles memory requirements. Furthermore, the tensor library is optimized for 1 OpenMP-thread per MPI rank, which also negatively impacts memory due to data duplication. Sparse tensor libraries are still an active field of research and need to be further investigated, as they will be crucial for developing novel AO based excited state methods.

To summarize, local excited state methods need a lot of know-how both in theoretical chemistry and programming. The road towards a truly robust local excited state method is arduous, but promising new approaches are proposed each year, and bring a new puzzle piece to the table.
