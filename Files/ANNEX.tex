\appendix
\chapter{Second Quantization}

Creation operator

One electron

two electron

\chapter{Hartree-Fock Starting Guesses \label{sec:SCFGUESS}}

\mchem{} can use three different starting guesses for the Hartree-Fock procedure: the core Hamiltonian, SAD and projection. This section gives some additional implementation details on the latter two methods

\section{Superposition of Atomic Densities}

The superposition of atomic densities (SAD) is a simple, yet powerful method to generate a HF starting guess that is already very close to the solution \cite{Van2006,Leh2019}. To a very good approximation, a molecule can be seen as a collection of atoms. The electronic guess density $\mbf{P}$ is then simply the direct sum of the individual atomic densities 
\begin{equation}
\mbf{P} = \bigoplus_{i=0}^{N_{atoms}} \mbf{P}^{atomic}_i 
\end{equation}
\noindent The resulting density matrix $\mbf{P}$ is block-diagonal. The atomic densities are obtained by performing a Hartree-Fock calculation on the individual atoms using partial occupation.

The SAD guess only gives an initial density, but no molecular orbitals, which might be necessary in some cases to construct the Fock matrix, depending on which algorithm is chosen. There are two choices in \mchem{} to generate starting orbitals:
\begin{itemize}
\item Natural orbitals by diagonalization of the guess density
\item Local molecular orbitals by performing an incomplete Cholesky decomposition with full pivoting on the guess density
\end{itemize}
The Cholesky decomposition has the advantage of revealing the rank of the matrix, and are therefore much more compact and the default option in \mchem{}

\subsection{Partial Occupation Hartree-Fock}

There are several different ways to perform the atomic Hartree-Fock calculations. Individual atoms often have  unpaired electrons, and may be computed using unrestricted Hartree-Fock. Alternatively, it is possible to perform fractional occupation Hartree-Fock (FOHF) calculations \cite{Bra1980}. After diagonalization of the Fock matrix, FOHF scales the coefficient matrices by an occupation vector $\mbf{v}$
\begin{equation}
\mbf{C}^{\sigma}_{frac} = \mbf{C}^{\sigma} \mbf{v}
\end{equation} 
\noindent where $\sigma$ is either $\alpha$ or $\beta$ spin. In standard HF, all entries in $\mbf{v}$ are set to 1 for occupied, and 0 for virtual orbitals. FOHF allows fractional values between 0 and 1 for occupied orbitals. The exact form of $\mbf{v}$ depends on the atom. Consider for example the oxygen atom with configuration 1s$^{\uparrow\downarrow}$2s$^{\uparrow\downarrow}$2p$_x^{\uparrow\downarrow}$2p$_y^{\uparrow}$2$_z^{\uparrow}$, using the aufbau principle. In standard UHF, the single $\downarrow$ (or $\beta$) electron in the p orbitals will occupy \emph{either} $x$, $y$ or $z$. However, in absence of any external perturbation, the electron should have no preference for any of them. FOHF allows the electron to occupy all orbitals, and the occupation vectors for the occupied AOs are given by
\begin{align}
\mbf{v}^{\alpha} &= \{1,1,1,1,1\} \\
\mbf{v}^{\beta} &= \{1,1,\frac{1}{3},\frac{1}{3},\frac{1}{3}\}
\end{align}
\noindent Through electron-delocalization across all three p-orbitals, the energy of the FOHF wave function is actually lowered compared to the UHF wave function. FOHF can substantially improve the description of single atoms and open-shell molecules. Restricted FOHF is also possible by simply spin-averaging the $\alpha$ and $\beta$ occupation vectors:
\begin{equation}
v^{restricted} = \frac{1}{2} \left( \mbf{v}^{\alpha} + \mbf{v}^{\beta} \right)
\end{equation}
\noindent \mchem{} supports both unrestricted and restricted FOHF, though only for a single atoms. The occupation numbers for each atom-type are hard-coded and taken from Psi4 \cite{Tur2012}.

\section{Projection Methods}

For Hartree-Fock calculations that encounter convergence difficulties when using large basis sets with many diffuse basis functions, it may be beneficial to first compute the HF wave function in a minimal basis, and then project it onto the larger basis set \cite{Leh2019}. Let $\sket{\chi^{min}}$ be the atomic orbitals for the minimal basis set and $\sket{\chi^{full}}$ the AOs of the larger basis set. By defining the projection operator for a non-orthogonal AO basis
\begin{equation}
\hat{P} = \sket{\chi_{\mu}} S_{\mu\nu}^{-1} \sbra{\chi_{\nu}}
\end{equation}
\noindent the MOs computed in the minimal basis set can then be projected onto the larger AO space as
\begin{equation}
\begin{split}
\hat{P}^{full}\sket{\phi_i} &= \hat{P}^{full}\sum_i C^{min}_{\mu i} \sket{\chi^{min}_\mu} \\
	&= (S^{full}_{\mu\nu})^{-1} \sbraket{\chi^{full}_{\nu}}{\chi^{min}_{\sigma}} \sket{\chi^{full}_{\sigma}}
\end{split} 
\end{equation}
\noindent Using the above expression, and introducing the cross-overlap matrix $S^{full,min} = \sbraket{\chi^{full}_{\nu}}{\chi^{min}_{\sigma}}$ between the two basis sets, the projected coefficient matrix is then computed as
\begin{equation}
\mbf{C}^{full} = (\mbf{S}^{full})^{-1} \mbf{S}^{full,min} \mbf{C}^{min} 
\end{equation}
\noindent The projection method is also useful in cases where the SAD guess cannot be used (see \ref{app:LINDEP}). 

\chapter{Removing Linear Dependencies in Basis Sets \label{app:LINDEP}}

Excited state methods often need larger basis sets with diffuse functions to accurately describe electron transitions into higher lying orbitals. However, large basis sets often introduce linear dependencies which may lead to decreased accuracy, numerical instabilities or even crashes. \mchem{} can use two different methods to remove linear dependencies from basis sets.

\section{Canonical Orthogonalization}

Most quantum chemistry programs use canonical orthogonalization \cite{Low1956,Low1970} to remove linear dependencies in basis sets. Diagonalization of the AO overlap matrix
\begin{equation}
\mbf{S} = \mbf{V} \boldsymbol{\Lambda} \mbf{V}\pdg
\end{equation}
\noindent gives the eigenvectors $\mbf{V}$ and the diagonal matrix $\boldsymbol{\Lambda}$ containing the eigenvalues. For near-linearly dependent basis sets, some the eigenvalues become very small and introduce large numerical errors for subsequent operations such as matrix inversion. Canonical orthogonalization removes all eigenvectors with eigenvalues below a certain threshold (1e-4 to 1e-6).

\section{Cholesky Decomposition}

Linear dependencies may alternatively be removed by an incomplete pivoted Cholesky decomposition of the overlap matrix \cite{Leh2019a} 
\begin{equation}
\mbf{PSP}^T = \mbf{L} \mbf{L}^T
\end{equation}
\noindent where $\mbf{P}$ is a permutation matrix, and $\mbf{L}$ are the Cholesky factors of dimension $N_{AO} \times r$, where $r$ is the rank of $\mbf{S}$. The Cholesky procedure as given in Algorithm \ref{algo:CHOLPIV}) outputs a vector $perm$ which contains the pivoting indices. A new basis set is constructed such that each atomic orbital with indices $perm[0:r]$ are included. The remaining functions $perm[r:N]$ are discarded as the pivots lie below the threshold. 

Because the removal of linear dependencies from basis sets can significantly alter the basis sets on individual atoms, the SAD method cannot be used for an initial guess, and projective approaches should be used.

%- ERI deomposition: cholesky, THC, pseudo-spectral
% cholesky https://link.springer.com/article/10.1007/s00214-009-0608-y
- The evil matrix inversion: considerations
% see https://www.johndcook.com/blog/2010/01/19/dont-invert-that-matrix/ 
% also https://epubs.siam.org/doi/abs/10.1137/1.9780898718027.ch14
%- mulliken, boughton pulay
%- Basis set overcompleteness