\chapter*{Abstract}

The algebraic diagrammatic construction (ADC) method, alongside coupled cluster linear response (CCLR) and equation-of-motion coupled cluster (EOMCC) are among the most accurate and predictive methods currently available for the calculation of excited state properties. However, even the most cost effective variants such as ADC(2) or the CC2 flavors of CCLR and EOMCC, still scale with the fifth power of the system size. In recent years, there has been an increased interest in local excited state methods, which borrow concepts from local correlation methods for computing ground state properties, to lower the scaling of canonical ADC, CCLR and EOMCC. By switching from the delocalized canonical molecular orbital (CMO) basis to a more spatially confined orbital representation, the computational complexity can be significantly lowered. Current implementations of local excited state method use local molecular orbitals (LMOs), natural orbitals (NOs), or combinations thereof. These methods often have the disadvantage of being state-specific, meaning that the compact orbital representation needs to recomputed for each individual excited state, which greatly increases the cost prefactor. Moreover, they introduce many parameters for controlling the construction of the orbitals, making them less robust.

In this thesis, a novel approach to local excited state methods is proposed, where the concept of the atomic orbital formulation of the second-order M{\o}ller-Plesset (MP2) energy expression is extended to ADC(2) by virtue of the Laplace transform (LT). The spin-opposite scaled second-order algebraic diagrammatic construction method with Cholesky decomposed densities and density fitting, or CDD-DF-SOS-ADC(2) for short, exploits the sparsity of the 2-electron repulsion integrals, the atomic ground state density matrix and the atomic transition density matrix to drastically reduce the computational effort. By using the local density fitting approximation, it is shown that linear scaling can be achieved for linear carboxylic acids. For electron-dense systems, near-quadratic scaling can still be achieved if the transition density is sparse, which is for example the case for hydrated formamide. Furthermore, the memory footprint and accuracy of the CDD-DF-SOS-ADC(2) method are explored in detail.

The CDD-DF-SOS-ADC(2) method is implemented in a new quantum chemistry software package called \mchem{}. It is MPI parallel and supports sparse matrix multiplication and tensor contraction through an interface to the distributed block compressed sparse row (DBCSR) library. The thesis discusses the implementation and structure of \mchem{} in detail, and summarizes the concepts of parallel computing, as well as the basics of matrix multiplication and storage formats.

\nonewpage

\selectlanguage{ngerman}
\chapter*{Zusammenfassung}

%The algebraic diagrammatic construction (ADC) method, alongside coupled luster linear response (CCLR) and the equation-of-motion coupled cluster (EOMCC) are among the most accurate and predictive methods currently available for the calculation of excited state methods. 
Das algebraisch-diagrammatische Konstruktionsschema (ADC), sowie die "{}Coupled Cluster Linear Response"{} (CCLR) und die "{}Equation-of-Motion Coupled Cluster"{} (EOM) Methoden zählen zu den exaktesten und prädikativisten Verfahren zur Berechnung von Eigenschaften von angeregten Zuständen, die aktuell zur Verfügung stehen. 
%However, even the most cost effective variants such as ADC(2) or the CC2 flavors of CCLR and EOMCC, still scale with the fifth power of the system size. 
Allerdings wachsen selbst die kostengünstigsten Varianten wie ADC(2), CC2LR oder EOM-CC2 mit der fünften Potenz der Systemgrö{\ss}e.
%In recent years, there has been an increased interest in local excited state methods, which borrow concepts from local correlation methods for computing ground state properties, to lower the scaling of canonical ADC, CCLR and EOMCC. 
In den letzten Jahren hat deshalb das Interesse an lokalen Methoden zur Berechnung angeregter Zustände stark zugenommen, welche Konzepte von lokalen Korrelationsmethoden für Grundzustände nutzen, um die Skalierung der Berechnungskosten von ADC, CCLR und EOMCC zu senken. 
%By switching from the delocalized canonical molecular orbital (CMO) basis to a more spatially confined orbital representation, the computational complexity can be significantly lowered. 
Die Rechenkomplexität kann signifikant reduziert werden, indem man von der delokalisierten kanonischen molekularen Orbitalbasis (CMO) zu einer neuen Basis wechselt, die stärker räumlich eingeschränkt ist. 
%Current implementations of local excited state method use local molecular orbitals (LMOs), natural orbitals (NOs), or combinations thereof. 
Die aktuellen Implementationen von lokalen Methoden im Rahmen angeregter Zustände nutzen meist lokale Molekularorbitale (LMO), natürliche Orbitale (NO) oder eine Kombination beider Repräsentationen.
%These methods often have the disadvantage of being state-specific, meaning that the compact orbital representation needs to recomputed for each individual excited state, which greatly increases the cost prefactor.
Allerdings haben LMO- und NO-Methoden oft den Nachteil, dass sie zustandsspezifisch sind, d.h. die kompakte Orbitalbasis muss für jeden einzelnen angeregten Zustand wieder berechnet werden, was den Vorfaktor der Rechnungskosten stark ansteigen lässt. 
%Moreover, they introduce many parameters for controlling the construction of the orbitals, making them somewhat less robust.
Au{\ss}erdem führen zusätzliche Parameter, die für die Konstruktion der Orbitale notwendig sind, dazu dass die Methoden weniger robust sind. 

%In this thesis, a novel approach to local excited state methods is proposed, where the concept of the atomic orbital formulation of the second-order M{\o}ller-Plesset (MP2) energy expression is extended to ADC(2) by virtue of the Laplace transform (LT). 
In dieser Dissertation wird eine neuartige lokale Methode zur Berechnung angeregter Zustände präsentiert, welche das Konzept einer Atomorbital-basierten Formulierung der M{\o}ller-Plesset Energie zweiter Ordnung durch die Laplace-Transformation auf die ADC(2) Methode erweitert.

%The spin-opposite scaled second-order algebraic diagrammatic construction method with Cholesky decomposed densities and density fitting, or CDD-DF-SOS-ADC(2) for short, exploits the sparsity of the 2-electron repulsion integrals, the atomic ground state density matrix and the atomic transition density matrix to drastically reduce the computational effort compared to canonical SOS-ADC(2). 
Die "{}Spin-Opposite Scaled"{} algebraisch-diagrammatische Konstruktionsmethode mit Cholesky-Zerlegung der Dichtematrizen und "{}Density-Fitting"{}, abgekürzt CDD-DF-SOS-ADC(2), nutzt die dünne Besetzungsstruktur der Zwei-Elektronen-Integrale, der elektronischen Dichtematrizen des Grundzustands und der Übergangsmatrizen des angeregten Zustands um die Rechenkomplexität im Vergleich zur kanonischen SOS-ADC(2) Methode stark zu reduzieren.
%By using the local density approximation, it is shown that linear scaling can be achieved for molecular systems such linear carboxylic acids. 
Mit lokalem "{}Density-Fitting"{} kann die Methode lineare Skalierbarkeit erreichen für molekulare Systeme wie lineare Carbonsäuren. 
%For electron-dense systems, near-quadratic scaling can still be achieved if the transition density is sparse. 
Selbst für Systeme mit hoher Elektronendichte skaliert die Methode fast mit der zweiten Potenz, wenn die Übergangsmatrix dünn besetzt ist, wie es z.B. für hydratisiertes Formamid der Fall ist.
%Furthermore, the memory footprint and accuracy of the CDD-DF-SOS-ADC(2) method are explored in detail.
Die Methode wird zudem auch auf Exaktheit und Speicherbedarf geprüft. 

%The CDD-DF-SOS-ADC(2) method is implemented in a new quantum chemistry software package called \mchem{}. 
Die CDD-DF-SOS-ADC(2) Methode ist in einem neuem Quantenchemiepaket namens \mchem{} implementiert.  
%It is MPI parallel and supports sparse matrix multiplication and tensor contraction through an interface to the distributed block compressed sparse row (DBCSR) library. 
Das Program ist MPI-parallel und unterstützt Multiplikation und Kontraktion von dünnbesetzten Matrizen und Tensoren durch die externe Programbibliothek DBCSR (Distributed Block-Compressed Sparse Row). 
%The thesis discusses the implementation in detail, and summarizes the concepts of parallel computing, as well as the basics of matrix multiplication and storage formats.
Die Implemtierung und die Struktur von \mchem{} werden im Detail beschrieben, und Grundlagen zu Matrix-Multiplikation, Matrix-Strukturen sowie die Funktionsweise von Parellelrechnern werden auch diskutiert.

\selectlanguage{american}

\nonewpage

