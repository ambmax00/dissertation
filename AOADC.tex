\chapter{The Spin-Opposite Scaled Algebraic Diagrammatic Construction Method in the Atomic Orbital Basis \label{cha:AOADC}}

The Algebraic Diagrammatic Construction method can be considered as M{\o}ller Plesset for excited states. It is therefore not surprising that local correlation methods for MP can also be applied to ADC. In chapter 4, it has been shown that local approximations for the ground state can be grouped into 3 categories: atomic orbitals, local orbitals and natural orbitals. Only the latter two have been used in the context of ADC as discussed in chapter 6. An atomic orbital representation of ADC has not yet been considered in literature, and will be the subject of this chapter. First, the restricted doubles-folded ADC working equations are derived. Then the SOS approximation is applied. Finally, the restricted SOS-ADC working equations are derived in the AO basis, with and without density fitting. 

\section{Working Equations for Restricted ADC with Doubles-Folding}

The eigenvalue problem in the algebraic diagrammatic construction method truncated at doubles excitations takes the form
\begin{equation}
\begin{bmatrix}
\mathbf{A}_{\mu_1\nu_1} & \mathbf{A}_{\mu_1\nu_2} \\
\mathbf{A}_{\mu_2\nu_1} & \mathbf{A}_{\mu_2\mu_2}
\end{bmatrix} 
\begin{bmatrix}
\mathbf{v}_{\mu_1} \\
\mathbf{v}_{\mu_2}
\end{bmatrix}
=  
\begin{bmatrix}
\mathbf{v}_{\mu_1} \\
\mathbf{v}_{\mu_2}
\end{bmatrix}
\mathbf{\Omega}
\end{equation}
\noindent where $\mathbf{A}$ is the symmetric ADC matrix with the singles-singles ($\mu_1\nu_1$), doubles-singles ($\mu_2\nu_1$), singles-doubles ($\mu_1\nu_2$) and doubles-doubles ($\mu_2\nu_2$) sub-blocks, with the eigenvectors $\mathbf{v}$ and the diagonal eigenvalue matrix $\boldsymbol{\Omega}$. The eigenvalue problem is generally solved using the Davidson diagonalization procedure to extract the first few lowest eigenvalues. Rather than constructing the entire Jacobian matrix which scales as $\ccpx{8}$, the Davidson method computes the matrix-vector products $\mathbf{r} = \mathbf{A} \mathbf{u}$ with the current trial vectors $\mathbf{u}$ using a closed-form expression, which reduces the computational complexity to $\ccpx{5}$. The MVPs can be expressed in block-form as
\begin{equation}
\begin{split}
\mathbf{r}_{\mu_1} = \mathbf{A}_{\mu_1\nu_1} \mathbf{u}_{\nu_1} + \mathbf{A}_{\mu_1\nu_2} \mathbf{u}_{\nu_2} \\
\mathbf{r}_{\mu_2} = \mathbf{A}_{\mu_2\nu_1} \mathbf{u}_{\nu_1} + \mathbf{A}_{\mu_2\nu_2} \mathbf{u}_{\nu_2}
\end{split} 
\end{equation}
\noindent The trial vector space in the Davidson diagonalization scales with fourth order as $o^2v^2$, and can quickly become a memory bottle-neck for large molecules. As shown in chapter 2, one can recompute the doubles-part of the MVP on-the-fly using \emph{doubles-folding}
\begin{equation}
\mathbf{r}_{\mu_1}(\omega) = \mathbf{A}_{\mu_1\nu_1} \mathbf{u}_{\nu_1} + \frac{\mathbf{A}_{\mu_1\zeta_2} \mbf{A}_{\zeta_2\nu_1}}{\omega - \mathbf{A}_{\zeta_2\zeta_2}} \mathbf{u}_{\nu_1} 
\end{equation} 
\noindent This trick is only possible for ADC(2)-s where the doubles-doubles block of the ADC matrix is diagonal. While the memory footprint for the diagonalization is reduced from $o^2v^2$ to $ov$, the MVP becomes dependent on the eigenvalue $\omega$ and a modified Davidson procedure needs to be used to solve this \emph{pseudo}-eigenvalue problem. The working equations for the folded ADC(2) matrix-vector product is given by \cite{Wor2009}
\begin{align}
\begin{split}
r_{IA} &= \sum_B u_{IB} f_{AB} - \sum_J f_{IJ} u_{JB} + \sum_{JB} \sbra{JA}\sket{IB} u_{JB} + \left[ \sum_B I^{(1)}_{AB} u_{IB} + \sum_J I^{(2)}_{IJ} u_{JA}\right] \\
&- \frac{1}{2} \left[ \sum_{JB} t_{IJAB} I^{(3)}_{JB} - \sum_{JB} \sbra{IJ}\sket{AB} I^{(4)}_{JB} \right] \\
&+ \left[\sum_{JKB} \sbra{JK}\sket{IB} u_{JKAB}(\omega) + \sum_{JBC} u_{IJBC}(\omega) \sbra{JA}\sket{BC}\right]
\end{split}
\end{align}
\noindent where the doubles part is computed as
\begin{align}
u_{IAJB}(\omega) &= \frac{1}{2} \frac{\left[\sum_K \sbra{IJ}\sket{KB}u_{KA} - \sbra{IJ}\sket{KA} u_{KB}\right] + \sum_C u_{IC} \sbra{JC}\sket{AB} - u_{JC} \sbra{IC}\sket{AB}}{\omega + \eps_I + \eps_J - \eps_A - \eps_B}
\end{align}
\noindent and with the intermediates $I^{(1)}$, $I^{(2)}$, $I^{(3)}$ and $I^{(4)}$ as given in \cite{Wor2009}. Until now, only spin molecular orbitals were assumed. Implementations such as adcman in Q-Chem \cite{Sha2006} can use these formulae directly by delegating any considerations of spin-symmetry to a special tensor library called libtensor \cite{Epi2013}, which programmatically keeps track of the non-vanishing spin block components and reduces the expressions to the restricted ADC(2) equations for closed-shell molecules.

If no special block tensor library is used, it is numerically advantageous to split the ADC(2) matrix-vector products into their spin-components, and compute each block individually. Using a double-bar notation to indicate MOs with opposite spin $\sigma(i) \neq \sigma(\ool{i})$, the matrix-vector product can be written as
\begin{equation}
\begin{split}
r_{ia}( \omega) = &(\eps_a - \eps_i) u_{ia} - \sum_{jb} \left[ \cn{ij}{ab} - \cn{ia}{jb} \right] u_{jb} + \sum_{\ool{jb}} \cn{ia}{\ool{jb}} u_{\ool{jb}} \\
&+ \sum_b I_{ab} u_{ib} + \sum_{j} I_{ij} u_{ja} - \frac{1}{2} \left[ t_{ia\ool{jb}} I^{(1)}_{\ool{jb}} + \cn{ia}{\ool{jb}} I_{\ool{jb}}^{(2)} \right] \\
&- \frac{1}{2} \left[ \left( t_{iajb} - t_{jaib} \right) I^{(1)}_{jb} + \left(\cn{ia}{jb} - \cn{ja}{ib} \right) I_{jb}^{(2)} \right] \\
&+ \sum_{kcl} u_{kalc}(\omega) \cn{ik}{cl} + \sum_{k\ool{cl}} u_{ka\ool{lc}}(\omega) \cn{ik}{\ool{cl}} \\
&- \sum_{ckd} \cn{ac}{kd} u_{ikcd}(\omega) - \sum_{c\ool{kd}} \cn{ac}{\ool{kd}} u_{ic\ool{kd}}(\omega)
\end{split}
\end{equation}  
\noindent with the preiteration intermediates (computed only once)
\begin{equation}
\begin{split}
I_{ab} &= \frac{1}{2} \sum_{kcl} \left[ t_{kalc} \cn{kb}{lc} - t_{kalc} \cn{kc}{lb} + \cn{ak}{cl} t_{kblc} - \cn{al}{ck} t_{kb}{lc} \right] \\
&+ \frac{1}{2} \sum_{k\ool{cl}} \left[ t_{ka\ool{lc}} \cn{kb}{\ool{lc}} + \cn{ak}{\ool{cl}} t_{kb}{\ool{lc}} \right] 
\end{split}
\end{equation}
\begin{equation}
\begin{split}
I_{ij} &= \frac{1}{2} \sum_{ckd} \left[ t_{ickd} \cn{jc}{kd} - t_{ickd} \cn{jd}{kc} + \cn{ci}{dk} t_{jckd} - \cn{ck}{di} t_{jckd} \right] \\
&+ \frac{1}{2} \sum_{c\ool{kd}} \left[ t_{ic\ool{kd}} \cn{jc}{\ool{kd}} + \cn{ci}{\ool{dk}} t_{jc\ool{kd}} \right]  
\end{split}
\end{equation}
\noindent and the iteration intermediates which depend on the trial vectors $\mathbf{u}$ (computed at each Davidson iteration)
\begin{equation}
I_{ia}^{(1)} = \sum_{jb} \left[ \cn{jb}{ia} - \cn{ja}{ib} \right] u_{jb} + \sum_{\ool{jb}} \cn{\ool{jb}}{ia} u_{jb}
\end{equation}
\begin{equation}
I_{ia}^{(2)} = \sum_{jb} \left[ t_{iajb} - t_{jaib} \right] u_{jb} + \sum_{\ool{jb}} t_{ia\ool{jb}} u_{\ool{jb}}
\end{equation}
\noindent The doubles spin components are computed on-the-fly and read
\begin{equation}
\begin{split}
u_{iajb}(\omega) &= \frac{1}{\omega - \eps_a - \eps_b + \eps_i + \eps_j} \left\lbrace \sum_k \left[ u_{ka} \cn{ki}{bj} - u_{ka} \cn{kj}{bi} - u_{kb} \cn{ki}{aj} \right. \right. \\
&+ \left. \left. u_{kb} \cn{kj}{ai} \right] - \sum_c \left[ u_{ic} \cn{ac}{bj} - u_{ic} \cn{aj}{bc} - u_{jc} \cn{ac}{bi} + u_{jc} \cn{bc}{ai} \right] \right\rbrace
\end{split}
\end{equation}
\begin{equation}
\begin{split}
u_{ia\ool{jb}}(\omega) = \frac{1}{\omega - \eps_a - \eps_{\olb} + \eps_i + \eps_{\olj}} \left\lbrace \sum_k u_{ka} \cn{ki}{\ool{bj}} + \sum_{\ool{k}} u_{\ool{kb}} \cn{\ool{kj}}{ai} \right. \\ - \sum_c u_ic \cn{ac}{\ool{bj}} - \left. \sum_{\ool{c}} u_{\ool{jc}} \cn{\ool{bc}}{ai} \right\rbrace
\end{split}
\end{equation}
\noindent The expression for the MVP for beta electrons ($r_{\ool{ia}})$ is obtained by replacing alpha orbitals ($i$) by beta orbitals ($\ool{i}$) and vice-versa in the expressions above. The off-diagonal blocks $r_{i\oola}$/$r_{\oli a}$ i.e. the spin-flip states will not be considered here and are set to zero. For closed-shell molecules, the complexity of the formulas can be drastically reduced by introducing the following spin-symmetry relationships:
\begin{align}
t_{iajb} &= t_{ia\ool{jb}} = t_{\ool{ia}jb} = t_{\ool{iajb}}   \label{eq:SPIN1}\\
\cn{ia}{jb} &= \cn{ia}{\ool{jb}} = \cn{\ool{ia}}{jb} = \cn{\ool{ia}}{\ool{jb}} \label{eq:SPIN2} \\
u_{ia} &= u_{\ool{ia}} \qquad \textrm{if singlet} \label{eq:SPIN3} \\
u_{ia} &= - u_{\ool{ia}} \qquad \textrm{if triplet} \label{eq:SPIN4}
\end{align}
\noindent One then obtains two separate expressions for restricted ADC(2), depending on whether singlet or triplet states are addressed
\begin{align}
\begin{split}
r_{ia}^S(\omega) &= (\eps_a - \eps_i) u^S_{ia} - \sum_{jb} \left[ 2\cn{ia}{jb} - \cn{ij}{ab} \right] u_{jb}^S + \sum_b I_{ab} u^S_{ib} + \sum_j u_{ja}^S I_{ij} \\
&- \frac{1}{2} \sum_{jb} \left[2t_{iajb} - t_{ibja}\right] I^{(1)S}_{jb} - \frac{1}{2} \sum_{jb} \left[ 2 \cn{ia}{jb} - \cn{ib}{ja} \right] I^{(2)S}_{jb} \\
&+ \sum_{kcl} \cn{ik}{lc} \left( 2 u_{kalc}^S(\omega) - u_{lakc}^S(\omega) \right) + \sum_{ckd} \left( 2 u^S_{ickd}(\omega) - u^S_{kcid}(\omega) \right) \cn{kd}{ac}  
\end{split}
\label{eq:SPINADCS}
\\
\begin{split}
r_{ia}^T(\omega) &= (\eps_a - \eps_i) u_{ia}^T - \sum_{jb} \cn{ij}{ab} u_{jb}^T + \sum_b I_{ab} u_{ib}^T + \sum_j I_{ij} u_{ja}^T \\
&+ \frac{1}{2} \sum_{jb} t_{ibja} I^{(1)T}_{jb} + \frac{1}{2} \sum_{jb} \cn{ib}{ja} I^{(2)T}_{jb} \\
+& \sum_{kcl} \cn{ik}{lc} u^T_{kalc}(\omega) + \sum_{ckd} \left[ 2 u^T_{ickd}(\omega) - u^T_{idkc}(\omega) - u^T_{kcid}(\omega) \right] 
\end{split}
\label{eq:SPINADCT}
\end{align}
\noindent with the singlet and triplet doubles intermediates
\begin{align}
\begin{split}
u^{S,SOS}_{iajb}(\omega) &= \frac{c_{osc}}{\omega - \eps_a + \eps_i - \eps_b + \eps_j} \left\lbrace \sum_k \left[ u^S_{ka} \cn{ki}{jb} + u^S_{kb} \cn{kj}{ai} \right] \right. \\
&\left. - \sum_c \left[ u^S_{ic} \cn{jb}{ac} - u^S_{jc} \cn{ib}{ac} \right] \right\rbrace
\end{split} 
\\
u^T_{iajb}(\omega) &= \frac{1}{\omega - \eps_a + \eps_i - \eps_b + \eps_j} \left\lbrace \sum_k u^T_{ka} \cn{ki}{bj} - \sum_c u^T_{ic} \cn{ac}{jb} \right\rbrace
\end{align}
\noindent The pre-iteration intermediates are given by
\begin{align}
\begin{split}
I_{ab} &= \frac{1}{2} \sum_{kcl} \left[ \left( 2t_{kalc} - t_{kcla}\right) \cn{kb}{lc} + \left( 2t_{kblc} - t_{kclb} \right) \cn{ka}{lc} \right] \\
&= \frac{1}{2} \left[ \sum_{kcl} \left( 2t_{kalc} - t_{kcla}\right) \cn{kb}{lc} \right]\sym{a}{b}
\end{split}
\\
\begin{split}
I_{ij} &= \frac{1}{2} \sum_{ckd} \left[ \left( 2t_{ickd} - t_{idkc}\right) \cn{jc}{kd} + \left( 2t_{jckd} - t_{jdkc} \right) \cn{ic}{kd} \right] \\
&= \frac{1}{2} \left[ \sum_{ckd} \left( 2t_{ickd} - t_{idkc}\right) \cn{jc}{kd} \right]\sym{i}{j}
\end{split}
\end{align}
\noindent and are the same for both singlet and triplet expressions. The iterative intermediates however are split:
\begin{align}
I^{(1)S}_{ia} &= \sum_{jb} \left( 2\cn{ia}{jb} - \cn{ib}{ja} \right) u_{jb}^S
\\
I^{(2)S}_{ia} &= \sum_{jb} \left( 2t_{iajb} - t_{ibja}\right) u_{jb}^S
\\
I^{(1)T}_{ia} &= - \sum_{jb} \cn{ib}{ja} u_{jb}^T
\\
I^{(2)T}_{ia} &= - \sum_{jb} t_{ibja} u_{jb}^T
\end{align}
%\noindent For the restricted equations ... to ..., the indices $ijkl...$ and $abcd...$  represent the occupied and virtual spin-integrated \emph{spatial} molecular orbitals. 

\section{Working Equations for Restricted SOS-ADC(2) with Doubles-Folding}

In the restricted ADC expressions for the matrix-vector product, the 4-index intermediates $u_{iajb}(\omega)$ need to be evaluated and temporarily stored, even if the density fitting approximation is used. This memory-intensive step can be avoided by using spin-opposite scaling (Section \ref{sec:SOSADC}). Consider again the approximations introduced by the SOS method for the unrestricted ADC(2) matrix equations
\begin{itemize}
\item In the spin-amplitudes $\hat{t}_{iajb}$ the same-spin contributions are nulled and the opposite-spin contributions are scales by $c_{os}$
\begin{equation}
\hat{t}_{SOS} = c_{os} \hat{t}_{iajb} \left( 1 - \delta_{\sigma (i) \sigma (j)} \right)
\end{equation}
\item In the doubles-singles and singles-doubles block of the ADC matrix, some same-spin contributions are also removed, and the whole block is scaled by a different constant $c_{osc}$ 
\begin{equation}
C_{ia,kcld} = c_{osc} \left[ \sbk{kl}{id} \delta_{ac} - \sbk{kl}{ic} \delta_{ac} - \sbk{al}{cd} \delta_{ik} + \sbk{ak}{cd}\delta_{il} \right] \left( 1 - \delta_{\sigma (k) \sigma (l)} \right)
\end{equation}
\begin{equation}
C_{iajb,kc} = c_{osc} \left[ \sbk{kb}{ij} \delta_{ac} - \sbk{ka}{ij} \delta_{bc} - \sbk{ab}{cj} \delta_{ik} + \sbk{ab}{ci}\delta_{jk} \right] \left( 1 - \delta_{\sigma (i) \sigma (j)} \right)
\end{equation}
\noindent Here, the function $\sigma(x)$ returns the spin of orbital $x$.
\end{itemize}
Applying the above constraints to the MVP expression \ref{eq:SPINADCS} (shown for singles only) gives the spin components of the SOS-ADC(2) matrix-vector product:
\begin{equation}
\begin{split}
r_{ia}^{SOS}( \omega) = &(\eps_a - \eps_i) u_{ia} - \sum_{jb} \left[ \cn{ij}{ab} - \cn{ia}{jb} \right] u_{jb} + \sum_{\ool{jb}} \cn{ia}{\ool{jb}} u_{\ool{jb}} \\
&+ \sum_b I^{SOS}_{ab} u_{ib} + \sum_{j} I^{SOS}_{ij} u_{ja} - \frac{1}{2} \left[ t_{ia\ool{jb}} I^{(1)SOS}_{\ool{jb}} + \cn{ia}{\ool{jb}} I_{\ool{jb}}^{(2)SOS} \right] \\
&- \frac{1}{2} \left[\cn{ia}{jb} - \cn{ja}{ib} \right] I_{jb}^{(2)SOS} \\
&+ c_{osc} \left\lbrace \sum_{k\ool{cl}} u_{ka\ool{lc}}(\omega) \cn{ik}{\ool{cl}} - \sum_{c\ool{kd}} \cn{ac}{\ool{kd}} u_{ic\ool{kd}}(\omega) \right\rbrace
\end{split}
\label{SPINSOSADC2S}
\end{equation}  
\noindent with the SOS pre-iteration intermediates 
\begin{align}
\begin{split}
I_{ab}^{SOS} &= \frac{1}{2} \sum_{k\ool{cl}} \left[ t_{ka\ool{lc}} \cn{kb}{\ool{lc}} + \cn{ak}{\ool{cl}} t_{kb\ool{lc}} \right]
\end{split}
\\
\begin{split}
I_{ij}^{SOS} &= \frac{1}{2} \sum_{c\ool{kd}} \left[ t_{ic\ool{kd}} \cn{jc}{\ool{kd}} + \cn{ci}{\ool{dk}} t_{jc\ool{kd}} \right]  
\end{split}
\end{align}
\noindent and the singlet SOS iteration intermediates
\begin{align}
I_{ia}^{(1)SOS} &= \sum_{jb} \left[ \cn{jb}{ia} - \cn{ja}{ib} \right] u_{jb} + \sum_{\ool{jb}} \cn{\ool{jb}}{ia} u_{\ool{jb}}
\\
I_{ia}^{(2)SOS} &= \sum_{\ool{jb}} t_{ia\ool{jb}} u_{\ool{jb}}
\end{align}
\noindent Only the opposite-spin components of the doubles components are needed:
\begin{equation}
\begin{split}
u_{ia\ool{jb}}^{SOS}(\omega) = \frac{c_{osc}}{\omega - \eps_a - \eps_b + \eps_i + \eps_j} \left\lbrace \sum_k u_{ka} \cn{ki}{\ool{bj}} + \sum_{\ool{k}} u_{\ool{kb}} \cn{\ool{kj}}{ai} \right. \\ - \sum_c u_ic \cn{ac}{\ool{bj}} - \left. \sum_{\ool{c}} u_{\ool{jc}} \cn{\ool{bc}}{ai} \right\rbrace
\end{split}
\end{equation}
\noindent Finally, for closed-shell molecules, the matrix vector product for singlet excitations for restricted SOS-ADC(2) can be obtained by inserting the spin-symmetry relationships \ref{eq:SPIN1} to \ref{eq:SPIN4} into Equation \ref{SPINSOSADC2S}, and performing a similar manipulation for the triplet expression:
%\begin{myframe}{EQUATION}
\begin{align}
\begin{split}
r_{ia}^{S,SOS}(\omega) &= (\eps_a - \eps_i) u_{ia}^S - \sum_{jb} \left[ 2\cn{ia}{jb} - \cn{ij}{ab} \right] u_{jb}^S + \sum_b I^{SOS}_{ab} u_{ib} + \sum_j u_{ja}^S I^{SOS}_{ij} \\
&- \frac{1}{2} \sum_{jb} t^{SOS}_{iajb}I^{(1)S,SOS}_{jb} - \frac{1}{2} \sum_{jb} \left[ 2\cn{ia}{jb} - \cn{ib}{ja} \right] I^{(2)S,SOS}_{jb} \\
&+ c_{osc} \left\lbrace \sum_{kcl} \cn{ik}{lc} u^{S,SOS}_{kalc}(\omega) - \sum_{ckd} u^{S,SOS}_{ickd}(\omega) \cn{kd}{ac} \right\rbrace
\end{split}
\label{eq:SOSRADC2S}
\\
\begin{split}
r_{ia}^{T,SOS}(\omega) &= (\eps_a - \eps_i) u_{ia}^S - \sum_{jb} \cn{ij}{ab} u_{jb}^T + \sum_b I^{SOS}_{ab} u_{ib} + \sum_j u_{ja}^S I^{SOS}_{ij} \\
&- \frac{1}{2} \sum_{jb} t^{SOS}_{iajb}I^{(1)T,SOS}_{jb} + \frac{1}{2} \sum_{jb} \cn{ib}{ja} I^{(2)T,SOS}_{jb} \\
&+ c_{osc} \left\lbrace \sum_{kcl} \cn{ik}{lc} u^{T,SOS}_{kalc}(\omega) - \sum_{ckd} u^{T,SOS}_{ickd}(\omega) \cn{kd}{ac} \right\rbrace
\end{split}
\label{eq:SOSRADC2T}
\end{align}
\noindent with the on-the-fly doubles
\begin{align}
\begin{split}
u_{iajb}^{S,SOS}(\omega) &= \frac{c_{osc}}{\omega - \eps_a - \eps_b + \eps_i + \eps_j} \left\lbrace \sum_k u_{ka} \cn{ki}{bj} + \sum_{k} u_{kb} \cn{kj}{ai} \right. \\ 
&- \sum_c u_ic \cn{ac}{bj} - \left. \sum_{c} u_{jc} \cn{bc}{ai} \right\rbrace
\end{split} \label{eq:SOSU2S}
\\
\begin{split}
u_{iajb}^{T,SOS}(\omega) &= \frac{c_{osc}}{\omega - \eps_a - \eps_b + \eps_i + \eps_j} \left\lbrace \sum_k u_{ka} \cn{ki}{bj} - \sum_{k} u_{kb} \cn{kj}{ai} \right. \\ 
&- \sum_c u_ic \cn{ac}{bj} + \left. \sum_{c} u_{jc} \cn{bc}{ai} \right\rbrace
\end{split} \label{eq:SOSU2T}
\end{align}
\noindent and the intermediates
\begin{align}
I_{ab}^{SOS} &= \frac{c_{os}}{2} \left[ \sum_{kcl} t_{kalc} \cn{kb}{lc} \right]\sym{a}{b}
\\
I_{ij}^{SOS} &= \frac{c_{os}}{2} \left[ \sum_{ckd} t_{ickd} \cn{jc}{kd} \right]\sym{i}{j}
\\
I^{(1)S,SOS}_{ia} &= \sum_{jb} \left( 2\cn{ia}{jb} - \cn{ib}{ja} \right) u_{jb}^S
\\
I^{(2)S,SOS}_{ia} &= c_{os} \sum_{jb} t_{iajb} u_{jb}^S
\\
I^{(1)T,SOS}_{ia} &= - \sum_{jb} \cn{ib}{ja} u_{jb}^T
\\
I^{(2)T,SOS}_{ia} &= - c_{os} \sum_{jb} t_{iajb} u_{jb}^T
\end{align}

\section[Working Equations For Restricted AO-SOS-ADC(2)]{Working Equations For Restricted SOS-ADC(2) with Doubles-Folding in an Atomic Orbital Basis}

The goal of an atomic orbital based formulation of ADC(2) is to compute the matrix-vector product in an intermediate AO basis and transform it back to the MO basis (or alternatively an LMO basis) for the Davidson procedure, similarly to how it is done for CIS:
\begin{equation}
r_{ia} = C_{\mu i} r_{\ulgm \olgn} C_{\nu a}
\end{equation}
\noindent Furthermore, it is convenient to split the MVP into six components which are evaluated individually
\begin{equation}
r_{ia}(\omega) = r_{ia}^{CIS} + r_{ia}^{2A} + r_{ia}^{2B} + r_{ia}^{2C} + r_{ia}^{2D} + r_{ia}^{2E}(\omega)
\end{equation}
\noindent In the next sections, using Equations \ref{eq:SOSRADC2S} and \ref{eq:SOSRADC2T} as starting points, the working equations for restricted AO-SOS-ADC(2) will be derived and discussed in detail. 

\subsection{First Order}

The first order part of the MVP is identical in both ADC(2) and SOS-ADC(2)
\begin{align}
r_{ia}^{S,CIS} &= (\eps_a - \eps_i) u^S_{ia} + \sum_{jb} \left[ 2\cn{ia}{jb} - \cn{ij}{ab}\right] u^S_{jb} \\
r_{ia}^{T,CIS} &= (\eps_a - \eps_i) u^T_{ia} - \sum_{jb} \cn{ij}{ab} u^T_{jb} 
\end{align}
\noindent An AO formulation is obtained in an identical manner to AO-CIS by factoring out the coefficient matrices to obtain Hartree-Fock-like expressions:
\begin{align}
\begin{split}
r_{ia}^{S,CIS,AO} &= (\eps_a - \eps_i) u^S_{ia} + \sum_{ia} C_{\mu i} C_{\sigma a} \left[ \left( 2\cn{\mu\sigma}{\nu\lambda} - \cn{\mu\nu}{\sigma\lambda} \right) u^S_{\ulgn \olgl} \right] \\
&= (\eps_a - \eps_i) u^S_{ia} + \sum_{ia} C_{\mu i} C_{\sigma a} \left[ 2\tilde{J}^S_{\mu\sigma} - \tilde{K}^S_{\mu\sigma} \right]
\end{split}
\\
\begin{split}
r_{ia}^{T,CIS,AO} &= (\eps_a - \eps_i) u^T_{ia} - \sum_{ia} C_{\mu i} C_{\sigma a} \left[ \cn{\mu\nu}{\sigma\lambda} u^T_{\ulgn \olgl} \right] \\
&= (\eps_a - \eps_i) u^T_{ia} - \sum_{ia} C_{\mu i} C_{\sigma a} \tilde{K}^T_{\mu\sigma} 
\end{split}
\end{align}
\noindent where $\mathbf{\tilde{J}}$ and $\mathbf{\tilde{K}}$ are the coulomb and exchange kernels, and $u_{\ulgm\olgs}$ is the transition density in the AO basis
\begin{equation}
u_{\ulgm\olgs} = C_{\mu i} u_{ia} C_{\olgs a}
\end{equation}
\noindent The zero order terms (i.e. the molecular orbital energy differences) do not need to be formulated in an AO basis, because the computation time is negligible. Similarly, transforming $\mathbf{\tilde{J}}$ and $\mathbf{\tilde{K}}$ to the MO basis formerly scales as $\ccpx{3}$ but has very low overhead and does not influence the overall scaling of AO-SOS-ADC(2). The time-determining steps are the computation of the J-kernel, which scales as $\ccpx{2}$ and the K-kernel, which scales as $\mathcal{O}(N)$ in the limit of large systems. For triplet excitations, the scaling is reduced to linear due to the absence of coulomb contributions.

\subsection{Second Order: Part 2A and 2B}

The expressions for component 2A and 2B read
\begin{align}
r_{ia}^{S,SOS,2A} = \sum_b I^{SOS}_{ab} u^S_{ib} + \sum_j I^{SOS}_{ij} u^S_{ja} \\
r_{ia}^{T,SOS,2A} = \sum_b I^{SOS}_{ab} u^T_{ia} + \sum_j I^{SOS}_{ij} u^T_{ja} 
\end{align}
\noindent with the intermediates as defined in the previous section. Rather than casting the whole expression into the AO basis, it is more convenient to evaluate only the non-symmetrized intermediates $I^{SOS,ns}_{ab}$ and $I^{SOS,ns}_{ij}$ in the AO basis.
The expressions for the intermediates involve the t-amplitudes, and to obtain an orbital-invariant formulation, it is necessary to use the Laplace transform
\begin{equation}
\frac{1}{\eps_a - \eps_i + \eps_b - \eps_j} = \sum^{nlap}_{\alpha} |w\pa| e^{-\eps_a t\pa} e^{\eps_i t\pa} e^{-\eps_b t\pa} e^{\eps_j t\pa} 
\end{equation}
\noindent Using a similar strategy to AO-MP2 to factor out the coefficient matrices, the intermediates can be formulated as
\begin{align}
I_{ab}^{AO-SOS,ns} &= \frac{c_os}{2} \sum_{kcl} \sum_{\alpha} |w\pa| e^{\Delta_{kalc}t\pa} \cn{ka}{lc} \cn{kb}{lc} \\
&= \frac{c_os}{2} \sum_b C_{\lambda b} \sum_{\alpha} \mid w\pa\mid^{1/4} e^{-\eps_a t\pa} C_{\sigma a} \sum_{\kappa \gamma \tau} \cn{\ulgk \sigma}{\ulgt \olgg}\pa \cn{\kappa \lambda}{\tau \gamma} \\
&= \frac{c_os}{2} \sum_b C_{\lambda b} \sum_{\alpha} \mid w\pa\mid^{1/4} e^{-\eps_a t\pa} C_{\sigma a} A\pa_{\sigma\lambda}
\end{align}
\noindent with the pseudo-AO electron integrals and the occupied/virtual pseudo density matrices
\begin{align}
\cn{\ulgk \sigma}{\ulgt \olgg}\pa &= P_{\kappa\kappa'}\pa \cn{\kappa' \sigma}{\tau' \gamma'} P_{\tau\tau'}\pa Q_{\gamma\gamma'}\pa \\
P_{\mu\mu'}\pa &= \sum_i C_{\mu i} e^{0.25 ln\wpa + \eps_i t\pa} C_{\mu' i} \\
Q_{\nu\nu'}\pa &= \sum_a C_{\nu a} e^{0.25 ln\wpa - \eps_a t\pa} C_{\nu' a}
\end{align}
Similarly
\begin{align}
I_{ij}^{AO-SOS,ns} &= \frac{c_{os}}{2} \sum_{\alpha} \sum_{ckd} \wpa e^{\Delta_{ickd}t\pa} \cn{ic}{kd} \cn{jc}{kd} \\
&= \frac{c_{os}}{2} \sum_j C_{\nu j} \sum_i \wpa^{1/4} C_{\mu i} e^{\eps_i t\pa} \sum_{\gamma\kappa\delta} \cn{\mu\olgg}{\ulgk\olgd} \cn{\nu\gamma}{\kappa\delta} \\
&= \frac{c_{os}}{2} \sum_j C_{\nu j} \sum_i \wpa^{1/4} C_{\mu i} e^{\eps_i t\pa} B_{\mu\nu}\pa
\end{align}
\noindent Finally, the intermediates are symmetrized
\begin{align}
I^{AO-SOS}_{ab} &= I^{AO-SOS,ns}_{ab} + I^{AO-SOS,ns}_{ba} \\
I^{AO-SOS}_{ij} &= I^{AO-SOS,ns}_{ij} + I^{AO-SOS,ns}_{ji}
\end{align}
The time-determining step for both intermediates is the computation of the Laplace intermediates $\mathbf{A}\pa$ and $\mathbf{B}\pa$. The subsequent multiplication with the coefficient matrices is again negligible. Consider now the sparsity diagram for the Laplace intermediate $\mathbf{A}\pa$:
\begin{center}
\begin{tikzpicture}

\snode{LAM}{1,0}{\lambda};
\snode{KAP}{2,0}{\kappa};
\snode{KAP2}{3,0}{\kappa'};
\snode{SIG}{4,0}{\sigma};
\snode{TAU}{5,0}{\tau};
\snode{TAU2}{6,0}{\tau'};
\snode{GAM2}{7,0}{\gamma'};
\snode{GAM}{8,0}{\gamma};

\draw[<->] (LAM) -- (KAP) node [midway, above] () {S};
\draw[<->] (KAP) -- (KAP2) node [midway, above] () {P};
\draw[<->] (KAP2) -- (SIG) node [midway, above] () {S};

\draw[<->] (TAU) -- (TAU2) node [midway, above] () {P};
\draw[<->] (TAU2) -- (GAM2) node [midway, above] () {S};
\draw[<->] (GAM2) -- (GAM) node [midway, above] () {P};

\end{tikzpicture}
\end{center}

\noindent The diagram has two edges, and hence $\mathbf{A}$ is evaluated with $\ccpx{2}$ computational complexity. The same can be shown for $\mathbf{B}$. The total memory footprint is also quadratic in $N$. 

\subsection{Second Order: Part 2C}

Component 2C is computed as
\begin{align}
r_{ia}^{S,SOS,2C} &= -\frac{c_{os}}{2} \sum_{jb} t_{iajb} I^{(1)S,SOS}_{jb} \\
r_{ia}^{T,SOS,2C} &= -\frac{c_{os}}{2} \sum_{jb} t_{iajb} I^{(1)T,SOS}_{jb}
\end{align}
\noindent Applying the Laplace transform, this then gives
\begin{align}
\begin{split}
r_{ia}^{S,AO-SOS,2C} &= -\frac{c_{os}}{2} \sum_{jb} \sum_{\alpha} \wpa e^{\Delta_{iajb}t\pa} \cn{ia}{jb} \left[\sum_{kc} \left(2\cn{jb}{kc} - \cn{jc}{kb} \right) u_{kc}^S \right] \\
&=  -\frac{c_{os}}{2} \sum_{\alpha} \sum_{ia} \wpa^{1/2} C_{\mu i} e^{\eps_i t\pa} C_{\sigma a} e^{-\eps_a t\pa} \\
&\left\lbrace \sum_{\nu\lambda} \cn{\mu\alpha}{\ulgn\olgl}\pa \left[ \sum_{\kappa\gamma}  \left( 2\cn{\nu\lambda}{\kappa\gamma} - \cn{\nu\gamma}{\kappa\lambda}\right) u^S_{\ulgk\olgg} \right] \right\rbrace \\
&=   -\frac{c_{os}}{2} \sum_{\alpha} \sum_{ia} \wpa^{1/2} C_{\mu i} e^{\eps_i t\pa} C_{\sigma a} e^{-\eps_a t\pa} \left\lbrace \sum_{\nu\lambda} \cn{\mu\alpha}{\ulgn\olgl}\pa \left[ 2\tilde{J}_{\lambda\nu} - \tilde{K}_{\lambda\nu} \right] \right\rbrace \\
&= -\frac{c_{os}}{2} \sum_{\alpha} \sum_{ia} \wpa^{1/2} C_{\mu i} e^{\eps_i t\pa} C_{\sigma a} e^{-\eps_a t\pa} I^{(1)(\alpha)S,AO-SOS}_{\mu \sigma}
\end{split} 
\end{align}
\noindent Similarly, triplet contributions are given by
\begin{align}
\begin{split}
r_{ia}^{T,AO-SOS,2C} &= \frac{c_{os}}{2} \sum_{alpha} \sum_{ia} \wpa^{1/2} C_{\mu i} e^{\eps_i t\pa} C_{\sigma a} e^{-\eps_a t\pa} \left\lbrace \sum_{\nu\lambda} \cn{\mu\alpha}{\ulgn\olgl}\pa \tilde{K}_{\lambda\nu} \right\rbrace \\
&= \frac{c_{os}}{2} \sum_{alpha} \sum_{ia} \wpa^{1/2} C_{\mu i} e^{\eps_i t\pa} C_{\sigma a} e^{-\eps_a t\pa} I^{(1)(\alpha)T,AO-SOS}_{\mu \sigma}
\end{split}
\end{align}
\noindent where $\mathbf{\tilde{J}}$ and $\mathbf{\tilde{K}}$ are the same matrices needed for the CIS contributions. Note that the matrices are \emph{transposed}, i.e. the index order is $\lambda\nu$, and not $\nu\lambda$. The time-determining step is the formation of the Laplace AO intermediates $I_{\mu\nu}^{\alpha(1)}$. Their sparsity diagrams read
\begin{center}
\begin{tikzpicture}

\snode{MU}{0,0}{\mu};
\snode{SIG}{1,0}{\sigma};
\snode{NU}{2,0}{\nu};
\snode{NU2}{3,0}{\nu'};
\snode{LAM2}{4,0}{\lambda'};
\snode{LAM}{5,0}{\lambda};
%\snode{INTER}{3.5,0}{};

\draw[<->] (MU) -- (SIG) node [midway, above] () {S};
\draw[<->] (NU) -- (NU2) node [midway, above] () {P};
\draw[<->] (NU2) -- (LAM2) node [midway, above] () {S};
\draw[<->] (LAM2) -- (LAM) node [midway, above] () {P};
\draw[<->] (NU) |- +(1.5,-0.5) node[below] {J/K} -| (LAM);

\end{tikzpicture}
\end{center}
\noindent The singlet and triplet AO intermediates are therefore evaluated in $\ccpx{2}$ time. 

\subsection{Second Order: Part 2D}

Now consider part 2D
\begin{align}
\begin{split}
r_{ia}^{S,SOS,2D} &= -\frac{1}{2}\sum_{jb} \left[ 2\cn{ia}{jb} - \cn{ib}{ja} \right] I_{jb}^{(2)S,SOS} \\
&= -\frac{1}{2}\sum_{jb} K_{iajb} I_{jb}^{(2)S,SOS}
\end{split}
\\
r_{ia}^{T,SOS,2D} &= \frac{1}{2}\sum_{jb} \cn{ib}{ja} I_{jb}^{(2)T,SOS}
\end{align}

\noindent Applying the Laplace transform gives the singlet expression
\begin{align}
\begin{split}
r_{ia}^{S,AO-SOS,2D} &= -\frac{c_{os}}{2} \sum_{jb} K_{iajb} \sum_{kc} \sum_{\alpha} \wpa e^{\Delta_{iajb}t\pa} \cn{jb}{kc} u_{kc}^S \\
&= -\frac{c_{os}}{2} \sum_{ia} C_{\mu i} C_{\sigma a} \left[  K_{\mu\sigma\nu\lambda} \left( \sum_{\alpha} \cn{\ulgn\olgl}{\kappa\gamma}\pa u^{(\alpha)S}_{\ulgk\olgg} \right)  \right] \\
&= -\frac{c_{os}}{2} \sum_{ia} C_{\mu i} C_{\sigma a} I^{(2)S,AO-SOS}_{\mu\sigma} 
\end{split} 
\end{align}
Similarly, the triplet expressions
\begin{align}
\begin{split}
r_{ia}^{T,AO-SOS,2D} &= \frac{c_{os}}{2} \sum_{jb} \cn{ia}{jb} \sum_{kc} \sum_{\alpha} \wpa e^{\Delta_{iajb}t\pa} \cn{jb}{kc} u_{kc}^T \\
&= \frac{c_{os}}{2} \sum_{ia} C_{\mu i} C_{\sigma a} \left[  \cn{\mu\sigma}{\nu\lambda} \left( \sum_{\alpha} \cn{\ulgn\olgl}{\kappa\gamma}\pa u^{(\alpha)T}_{\ulgk\olgg} \right) \right] \\
&= \frac{c_{os}}{2} \sum_{ia} C_{\mu i} C_{\sigma a} I^{(2)T,AO-SOS}_{\mu\sigma} 
\end{split}
\end{align}
\noindent With the transition density in the pseudo atomic orbital basis
\begin{equation}
u\pa_{\ulgm\olgs} = \wpa^{1/2} C_{\mu i} e^{\eps_i t\pa}  u_{ia} C_{\sigma a} e^{-\eps_a t\pa} 
\end{equation}
\noindent The computation of the AO intermediates $I^{(2)SOS-AO}$ is the time-determining step, and is best evaluated as
\begin{align}
\tilde{J}\pa_{\mu\sigma} &= \cn{\mu\sigma}{\nu\lambda} u\pa_{\ulgn\olgl} \\
\tilde{J}_{\ulgm\olgs}\pa &= P_{\mu\mu'}\pa \tilde{J}\pa_{\mu\nu} Q_{\nu\nu'}\pa \\
I^{(2)SOS-AO}_{\mu\sigma} &= \sum_{\alpha} \left[2 \cn{\mu\sigma}{\nu\lambda} - \cn{\mu\lambda}{\nu\sigma} \right] \tilde{J}_{\ulgn\olgl}\pa
\end{align}
\noindent Every individual step can be computed with $\ccpx{2}$ complexity, meaning the intermediate is also evaluated with overall quadratic effort.

\subsection{Second Order: Part 2E}

The final part is given by
\begin{align}
r_{ia}^{S,SOS,2E}(\omega) &= c_{osc} \left\lbrace \sum_{kcl} \cn{ik}{lc} u^{S,SOS}_{kalc}(\omega) - \sum_{ckd} u^{S,SOS}_{ickd}(\omega) \cn{kd}{ac} \right\rbrace  \label{eq:SIGMA2ES}\\
r_{ia}^{T,SOS,2E}(\omega) &= c_{osc} \left\lbrace \sum_{kcl} \cn{ik}{lc} u^{T,SOS}_{kalc}(\omega) - \sum_{ckd} u^{T,SOS}_{ickd}(\omega) \cn{kd}{ac} \right\rbrace
\label{eq:SIGMA2ET}
\end{align}
\noindent With the doubles intermediates as given in Equation \ref{eq:SOSU2S} and \ref{eq:SOSU2T}. The Laplace transform needs to be applied to the energy denominator present in these intermediates. The optimal Laplace parameters are however different from the ones used for the t-amplitudes, due to the additional factor of the excitation energy $\omega$. For each different excitation energy $\omega$, a new Laplace quadrature needs to be computed, alongside a new set of pseudo-density matrices $\mathbf{P}$ and $\mathbf{Q}$. The additional time is however negligible for the standard number of quadrature points ($n_{lap}$ $<$ 10). The symbol $\theta$ is used to designate the Laplace quadrature for the doubles denominator to differentiate them from the ones for the t-amplitudes. 

First, an AO formulation of the doubles amplitudes will be derived such that
\begin{equation}
u_{iajb}(\omega) = C_{\mu i} C_{\sigma a} u_{\mu\sigma\nu\lambda} C_{\nu j} C_{\lambda b}
\end{equation}
\noindent For quantities like the MO integrals $\cn{ia}{jb}$, this is straight-forwardly done by factoring out the coefficient matrices. However, the situation is more complex in the doubles intermediates, due to the presence of terms like $u_{ka}\cn{ki}{bj}$. For the MO transition densities, the non-orthogonality of the AO basis needs to be taken into consideration. The MO coefficient matrices are factored out by a PAO backtransform:
\begin{equation}
u_{ia} = C_{\mu i} S_{\mu\mu'} u_{\ulgm'\olgs'} S_{\sigma'\sigma} C_{\sigma a}
\end{equation}
The doubles intermediates can then be expressed as
\begin{equation}
\begin{split}
u_{iajb}^{S/T}(\omega) &= - c_{osc} \sum_{\theta} \sum_{\mu\sigma\nu\lambda} |w\pt| e^{\left(\omega - \eps_a - \eps_b + \eps_i + \eps_j\right)t\pt} C_{\mu i} C_{\sigma a} C_{\nu j} C_{\lambda b} \left\lbrace \vphantom{\sum_i} \right. \\
&\left. \sum_{\kappa} \left[ u^{S/T}_{\ulgk\olgs'} S_{\sigma'\sigma} \cn{\kappa \mu}{\nu\lambda} \pm u^{S/T}_{\ulgk\olgl'} S_{\lambda'\lambda} \cn{\nu\kappa}{\mu\sigma} \right] \right. \\
&\left. - \sum_{\gamma} \left[ S_{\mu\mu'} u^{S/T}_{\ulgm'\olgs} \cn{\nu\lambda}{\sigma\gamma} \pm S_{\nu\nu'} u^{S/T}_{\ulgn'\olgg} \cn{\mu\sigma}{\gamma\lambda} \right]   \right\rbrace \\
&= - \sum_{\theta} \sum_{\mu\sigma\nu\lambda} |w\pt| e^{\left(\omega - \eps_a - \eps_b + \eps_i + \eps_j\right)t\pt} C_{\mu i} C_{\sigma a} u_{\mu\sigma\nu\lambda}^{S/T} C_{\nu j} C_{\lambda b}
\end{split}
\label{eq:AODOUBLES}
\end{equation}
\noindent Note the additional minus sign in front of the Laplace summation. After the Laplace transform, the sign of the denominator is swapped, i.e. $\frac{1}{\pm x} \rightarrow exp(\mp x t\pt)$. For large negative occupied molecular orbital energies $\eps_i$ or large positive virtual molecular orbital energies $\eps_a$, this would lead to very large values and numerical instabilities. For this reason, the minus sign is factored out to reverse the sign in the exponent. 

Inserting \ref{eq:AODOUBLES} into Equations \ref{eq:SIGMA2ES} and \ref{eq:SIGMA2ET} gives the expression for part 2E constructed via AO intermediates:
\begin{equation}
\begin{split}
r_{ia}^{S/T,AO-SOS,2E}(\omega) &= - c_{osc}^2 \sum_{\theta} e^{\omega t\pt} \left\lbrace C_{\mu i} |w\pt|^{1/4} C_{\sigma a} e^{-\eps_a t\pt} \left[ \sum_{\kappa\gamma\tau} \cn{\mu\ulgk}{\ulgt\olgg}\pt u_{\kappa\sigma\tau\gamma}^{S/T} \right] \right. \\
& \left. - |w\pt|^{1/4} C_{\mu i} e^{\eps_i t\pt} C_{\sigma a} \left[\sum_{\gamma\kappa\delta} u_{\mu\gamma\kappa\delta}^{S/T} \cn{\ulgk\olgd}{\sigma\olgg}\pt \right] \right\rbrace \\
&= - c_{osc}^2 \sum_{\theta} e^{\omega t\pt} \left\lbrace C_{\mu i} |w\pt|^{1/4} C_{\sigma a} e^{-\eps_a t\pt} R^{(\theta)(1)S/T}_{\mu\olgs} \right. \\
& \left. - |w\pt|^{1/4} C_{\mu i} e^{\eps_i t\pt} C_{\sigma a} R^{(\theta)(2)S/T}_{\ulgm\sigma} \right\rbrace
\end{split}
\label{eq:SIG2EAO1}
\end{equation}
\noindent Similarly to previous expressions, the AO electron repulsion integrals are not completely transformed into the pseudo-AO basis, but only three-quarter transformed integrals are obtained. To obtain fully-transformed integrals, it is beneficial to perform the following transformation:
\begin{equation}
r_{ia} =  \bar{C}_{\mu i'} r_{\mu \sigma} \bar{C}_{\sigma a'} = \bar{C}_{\mu i'} C_{\mu i} r_{ia} C_{\sigma a} \bar{C}_{\sigma a'}
\end{equation} 
\noindent Inserting this expression into Equation \ref{eq:SIG2EAO1} yields
\begin{equation}
\begin{split}
r_{ia}^{S/T,AO-SOS,2E}(\omega) &= - c_{osc}^2 \sum_{\theta} e^{\omega t\pt} \left\lbrace \bar{C}_{\mu i} P_{\mu \nu} \bar{C}_{\sigma a} \left[ \sum_{\kappa\gamma\tau} \cn{\ulgn\ulgk}{\ulgt\olgg}\pt u_{\kappa\sigma\tau\gamma}^{S/T} \right] \right. \\
& \left. - \bar{C}_{\mu i} \bar{C}_{\sigma a} Q_{\sigma\nu} \left[\sum_{\gamma\kappa\delta} u_{\mu\gamma\kappa\delta}^{S/T} \cn{\ulgk\olgd}{\olgn\olgg}\pt \right] \right\rbrace \\
&= - c_{osc}^2 \sum_{\theta} e^{\omega t\pt} \left\lbrace \bar{C}_{\mu i} \bar{C}_{\sigma a} R^{(\theta)(1)S/T}_{\ulgm\olgs} - \bar{C}_{\mu i} \bar{C}_{\sigma a} R^{(\theta)(2)S/T}_{\ulgm\olgs} \right\rbrace 
\end{split}
\end{equation} 
\noindent which gives fully transformed integrals. This step is necessary to obtain a better factorization for part 2E in the density fitting approximation. Note that there are other cases of non-fully transformed integrals in the previous parts - however a full transformation does not give any significant advantage for a DF formulation, so they are left unchanged.  

The time-determining step is the formation of the $\mathbf{R}$ intermediates, which in turn depend on the AO doubles intermediates. Consider the sparsity diagram for the following term encountered in Equation \ref{eq:AODOUBLES}:

\begin{equation}
u_{\ulgk\olgs'} S_{\sigma'\sigma} \cn{\kappa \mu}{\nu\lambda}
\end{equation}

\begin{center}
\begin{tikzpicture}
\snode{NU}{-3,0}{\nu};
\snode{LAM}{-2,0}{\lambda};
\snode{MU}{-1,0}{\mu};
\snode{KAP}{0,0}{\kappa};
\snode{SIG1}{1,0}{\sigma'};
\snode{SIG}{2,0}{\sigma};

\draw[<->] (KAP) -- (SIG1) node [midway, above] () {P};
\draw[<->] (SIG1) -- (SIG) node [midway, above] () {S};
\draw[<->] (MU) -- (KAP) node [midway, above] () {S};

\draw[<->] (NU) -- (LAM) node [midway, above] () {S};

\end{tikzpicture}
\end{center}
\noindent Similar diagrams can be derived for the other three contractions in Equation \ref{eq:AODOUBLES}. This shows an overall quadratic scaling in computational effort and number of non-zero elements for the AO doubles intermediates $u_{\mu\sigma\nu\lambda}$. The indices $\mu$/$\sigma$ and $\nu$/$\lambda$ are connected by either an S/P junction, but no sparsity relationship can be established between those pairs, similar to the AO electron integrals.  

This information can be used to find the scaling of the $\mathbf{R}$ intermediates. For example, the sparsity diagram for $\mathbf{R}_{\ulgm\olgs}^{(1)} = \cn{\ulgm\ulgk}{\ulgt\olgg}\pt u_{\kappa\sigma\tau\gamma}$ reads
\begin{center}
\begin{tikzpicture}
\snode{MU}{-4,0}{\mu};
\snode{MU'}{-3,0}{\mu'};
\snode{KAP'}{-2,0}{\kappa'};
\snode{KAP}{-1,0}{\kappa};
\snode{SIG}{0,0}{\sigma};
\snode{TAU}{1,0}{\tau};
\snode{TAU1}{2,0}{\tau'};
\snode{GAM1}{3,0}{\gamma'};
\snode{GAM}{4,0}{\gamma};

\draw[<->] (MU) -- (MU') node [midway, above] () {P};
\draw[<->] (MU') -- (KAP') node [midway, above] () {S};
\draw[<->] (KAP') -- (KAP) node [midway, above] () {P};
\draw[<->] (KAP) -- (SIG) node [midway, above] () {S/P};
\draw[<->] (TAU) -- (TAU1) node [midway, above] () {P};
\draw[<->] (TAU1) -- (GAM1) node [midway, above] () {S};
\draw[<->] (GAM1) -- (GAM) node [midway, above] () {P};


\end{tikzpicture}
\end{center}
\noindent and a similar diagram can be drawn for $\mathbf{R}^{(2)}$, which again shows quadratic scaling. 

\subsection{Summary}

The AO-SOS-ADC(2) matrix-vector product is finally computed as
\begin{align}
\begin{split}
r_{ia}^{S,AO-DF-SOS}(\omega) &= (\eps_a - \eps_i) u_{ia}^S + \sum_{\mu\nu} C_{\mu i} C_{\nu a} \left( 2\tilde{J}^S_{\mu\nu} - \tilde{K}^S_{\mu\nu} \right) \\
&+ \sum_b I_{ab}^{AO-DF-SOS} u_{ib}^S + \sum_j I_{ij}^{AO-DF-SOS} u_{ja}^S \\
&- \frac{c_{os}}{2} \sum_{\alpha} \sum_{ia} \wpa^{1/2} C_{\mu i}\pa C_{\sigma a}\pa I_{\mu\sigma}^{(1)(\alpha)S,AO-DF-SOS} \\
&- \frac{c_{os}}{2} \sum_{ia} C_{\mu i} C_{\sigma a} I_{\mu\sigma}^{(2)S,AO-DF-SOS} \\
&- c_{osc}^2 \sum_{\theta} e^{\omega t\pt} \left\lbrace \bar{C}_{\mu i} \bar{C}_{\sigma a} R^{(\theta)(1)S}_{\ulgm\olgs} - \bar{C}_{\mu i} \bar{C}_{\sigma a} R^{(\theta)(2)S}_{\ulgm\olgs} \right\rbrace
\end{split}
\\
\begin{split}
r_{ia}^{T,AO-DF-SOS}(\omega) &= (\eps_a - \eps_i) u_{ia}^T + \sum_{\mu\nu} C_{\mu i} C_{\nu a} \tilde{K}^T_{\mu\nu} \\
&+ \sum_b I_{ab}^{AO-DF-SOS} u_{ib}^T + \sum_j I_{ij}^{AO-DF-SOS} u_{ja}^T \\
&- \frac{c_{os}}{2} \sum_{\alpha} \sum_{ia} \wpa^{1/2} C_{\mu i}\pa C_{\sigma a}\pa I_{\mu\sigma}^{(1)(\alpha)T,AO-DF-SOS} \\
&- \frac{c_{os}}{2} \sum_{ia} C_{\mu i} C_{\sigma a} I_{\mu\sigma}^{(2)T,AO-DF-SOS} \\
& - c_{osc}^2 \sum_{\theta} e^{\omega t\pt} \left\lbrace \bar{C}_{\mu i} \bar{C}_{\sigma a} R^{(\theta)(1)T}_{\ulgm\olgs} - \bar{C}_{\mu i} \bar{C}_{\sigma a} R^{(\theta)(2)T}_{\ulgm\olgs} \right\rbrace
\end{split}
\end{align}
\noindent where the intermediates are evaluated as presented in the previous sections. The 2-index intermediates still need to be transformed back to the canonical MO basis for the Davidson procedure, but the computational effort required is negligible. The overall cost of the AO-SOS-ADC(2) scales quadratically both in time and memory requirements.

It is expected that AO-SOS-ADC(2) has the same drawback as LinK or AO-MP2, namely a late onset of the low-scaling regime for larger basis sets. This additional overhead is even worse for AO-ADC due to the complexity of the formulas which involve many more tensor contractions than the ground state, and more S and P junctions which makes the method much more dependent on basis set size. This is further aggravated by the fact that diffuse basis functions are essential to obtain accurate excitation energies as opposed to ground state correlation energies.

\section{Working Equations for Restricted AO-DF-SOS-ADC(2) with Doubles-Folding}

To lower the steep scaling associated with increasing basis set size, the density fitting approximation is introduced. The two-electron repulsion integrals are approximated using the generalized form
\begin{equation}
\cn{\mu\sigma}{\nu\lambda} = B_{\mu\sigma X} M_{XY} B_{Y \nu\lambda}
\end{equation}
\noindent where the quantities $\mathbf{B}$ and $\mathbf{M}$ depend on the density fitting method. Furthermore, the J-,K- and Z-kernels are introduced:
\begin{align}
\calJ\left\lbrace M,P \right\rbrace_{\mu\nu} &= B_{\mu\nu X} M_{XY} B_{Y\sigma\lambda} P_{\lambda\sigma} \\
\calK\left\lbrace M,P \right\rbrace_{\mu\nu} &= B_{\mu\sigma X} M_{XY} B_{Y \nu\lambda} P_{\lambda\sigma} \\
\calZ\left\lbrace P,Q \right\rbrace_{XY} &= B_{X\mu\nu} P_{\mu\mu'} B_{\mu'\nu' Y} Q_{\nu' \nu} 
\end{align}
\noindent This notation allows to reduce the complexity of the formulas to some degree. 

The working equations for AO-DF-SOS-ADC(2) are given in Tables \ref{AODFSOSADC2INTERMEDIATES}, \ref{AODFSOSADC21D} and \ref{AODFSOSADC2E}. Table \ref{AODFSOSADC2INTERMEDIATES} shows the preiteration steps, that is, the computation of the Laplace parameters and the intermediates $I_{ij}$ and $I_{ab}$. The construction of the intermediates can be formulated in terms of the Z- and K-kernels. In the K-kernel, the metric matrix $\mbf{M}$ is replaced by the Laplace matrix $\mbf{G}$ formed by the Z-kernel. The contraction of $\mbf{G}$ with $B_{X\mu\nu}$ in the K-kernel is the most expensive step in the preiteration procedure. The resulting tensor can be computed on-the-fly as it is not needed for any other contraction.

Table \ref{AODFSOSADC21D} shows the steps to form part 1 and 2A-2D of the matrix-vector product. The steps are listed for both singlet and triplet. First the CIS Fock-like matrix $\mbf{F}^{CIS}$ is formed, as it represents an important intermediate for the subsequent steps. This is easily done using the J/K-kernels. Zero and first order contributions, as well as part 2A and 2B of the second order contributions are formed trivially afterwards using the intermediates. Part 2C is most conveniently computed using the J-kernel where the density matrix is replaced by a pseudo-density matrix $\mbf{H}$ formed by contracting with a pseudo-AO CIS matrix. Part 2D is a bit more involved. First, the matrix $\mbf{T}$ is formed by looping over the Laplace points and using the J-kernel. Then, this matrix is used in a Fock-like construction scheme to get the final result.

Finally, Table \ref{AODFSOSADC2E} shows the construction of part 2E of the MVP. This is the most expensive step of the whole procedure. First, the Laplace parameters are (re-)computed for the current excitation energy $\omega$. The algorithm then enters the Laplace loop in step 2. First, the intermediate tensors $B_{X\ulgm\olgn}$ and $R_{X\ulgm\ulgn}$ are formed and stored. They are then used to compute two intermediate matrices in the auxiliary basis, and the final intermediate $D_{X\ulgm\olgn}$ is formed. After two major exchange-like contraction steps, the contributions are added to part 2E. 

Using the density fitting approximation, the four-index intermediates $u_{\mu\sigma\nu\lambda}$ from AO-SOS-ADC(2) can be avoided, and the procedure only depends on 3-index intermediates. 

\SetKwBlock{Substep}{}{}

\begin{algorithm}
\setstretch{1.3}
Compute Laplace quadrature parameters $\{w\pa,t\pa\}$ for $(\eps_i + \eps_j - \eps_a - \eps_b)^{-1}$
\\
If needed by the J,K or Z kernels, compute the Cholesky decompositions of the occupied and virtual pseudo-density matrix 
Compute the intermediate matrices $I_{ij}$ and $I_{ab}$
\\
\For{\forcond}{
	$G\pa_{XY} \leftarrow M_{XR} \calZ\left\lbrace P\pa,Q\pa\right\rbrace_{RS} M_{SR}$
	\\
	$I_{ab} \leftarrow - \frac{c_{os}}{2} C_{\nu b} \sum_\alpha C_{\mu a} \mid w\pa \mid^{1/4} e^{-\eps_a t\pa} \calK\left\lbrace P\pa,G\pa\right\rbrace_{\mu\nu}$
	\\
	$I_{ij} \leftarrow - \frac{c_{os}}{2} C_{\nu j} \sum_\alpha C_{\mu i} \mid w\pa \mid^{1/4} e^{ \eps_i t\pa} \calK\left\lbrace Q\pa,G\pa\right\rbrace_{\mu\nu}$ 
}
\Substep(\textbf{Symmetrize matrices}){
$I_{ij} \leftarrow I_{ji}$
\\ 
$I_{ab} \leftarrow I_{ba}$
}
\caption{Pre-iterative steps for computing the AO-ADC(2) intermediates}
\label{AODFSOSADC2INTERMEDIATES}
\end{algorithm}
%
\begin{algorithm}
\setstretch{1.3}
\Substep(\textbf{Compute the CIS Fock matrices}){
if singlet: $U_{\mu\nu} \leftarrow C_{\mu i} u^S_{ia} C_{\nu a}$
\\
if triplet: $U_{\mu\nu} \leftarrow C_{\mu i} u^T_{ia} C_{\nu a}$
\\
if singlet: $F^{CIS}_{\mu\nu} \leftarrow 2 * \calJ\left\lbrace U,M \right\rbrace_{\nu\mu} - \calK\left\lbrace U,M \right\rbrace_{\nu\mu}$
\\
if triplet: $F^{CIS}_{\mu\nu} \leftarrow - \calK\left\lbrace U,M \right\rbrace_{\nu\mu}$
}

\Substep(\textbf{Add zero- and first-order terms}){
$r_{ia} \leftarrow (\eps_a - \eps_i) u_{ia}$
\\
$r_{ia} \leftarrow C_{\mu i} F^{CIS}_{\mu\nu} C_{\nu a}$
}
%
\Substep(\textbf{Compute part (A) and (B) of second-order term}){
$r_{ia} \leftarrow u_{ib} I_{ab}$
\\
$r_{ia} \leftarrow u_{ja} I_{ij}$
}
%
\Substep(\textbf{Compute part (C) of second-order term}){
\For{\forcond}{
if singlet: $H\pa_{\mu\nu} \leftarrow P\pa_{\mu'\mu} Q\pa_{\nu'\nu} F^{CIS}_{\nu'\mu'}$
\\
if triplet: $H\pa_{\mu\nu} \leftarrow - P\pa_{\mu'\mu} Q\pa_{\nu'\nu} F^{CIS}_{\nu'\mu'}$
\\
$r_{ia} \leftarrow \frac{-c_{os}}{4} \mid w\pa \mid^{1/2} C_{\mu i} e^{\eps_i t\pa} C_{\nu a} e^{-\eps_a t\pa}  \calJ\left\lbrace H\pa,M \right\rbrace_{\mu\nu}$
} % end for
} % end sub
%
\Substep(\textbf{Compute part (D) of second-order term}){
\For{\forcond}{
if singlet: $U\pa_{\mu\nu} \leftarrow \mid w\pa \mid^{1/2} C_{\mu i} e^{\eps_i t\pa} C_{\nu a} e^{-\eps_a t\pa} u^S_{ia}$
\\
if triplet: $U\pa_{\mu\nu} \leftarrow - \mid w\pa \mid^{1/2} C_{\mu i} e^{\eps_i t\pa} C_{\nu a} e^{-\eps_a t\pa} u^T_{ia}$
\\
$T_{\mu\nu} \leftarrow \frac{1}{2} \sum_{\alpha} P\pa_{\mu\mu'} Q\pa_{\nu\nu'} \calJ\left\lbrace U\pa,M \right\rbrace_{\nu\mu}$
}
%
$r_{ia} \leftarrow -\frac{c_{os}}{2} C_{\mu i} C_{\nu a} \left[ 2\calJ\left\lbrace T,M \right\rbrace_{\mu\nu} - \calK\left\lbrace T,M \right\rbrace_{\mu\nu} \right]$
} % end substep
\caption{Steps for computing the singles part of the MVP of AO-ADC(2)}
\label{AODFSOSADC21D}
\end{algorithm}

\begin{algorithm}
\setstretch{1.3}
Compute Laplace quadrature parameters $\{w\pt,t\pt\}$ for $(-\omega + \eps_i + \eps_j - \eps_a - \eps_b)^{-1}$ \\
\For{\forcond}{
Compute doubles pseudo-matrices $P\pt$ and $Q\pt$
\\
$B_{X \ulgm \olgs}\pt \leftarrow P\pt_{\mu\mu'} B_{X \mu'\nu'} Q\pt_{\nu'\nu}$
\\
$v_{\ulgm\olgs}^{(1)(\theta)} \leftarrow P\pt_{\mu\lambda} S_{\lambda\nu} u_{\ulgn\olgs}$
\\
$v_{\ulgm\olgs}^{(2)(\theta)} \leftarrow u_{\ulgm\olgg} S_{\gamma\lambda} Q\pt_{\lambda\sigma}$		
\\
$R_{X \ulgm \olgs}\pt \leftarrow P_{\mu\lambda}\pt B_{X \lambda\nu} v_{\ulgn\olgs}^{(1)(\theta)} - v_{\ulgm\olgg}^{(2)(\theta)} B_{X \gamma\nu} Q\pt_{\nu\sigma}$
\\
$H_{XY}\pt \leftarrow B_{X \ulgm\olgs}\pt B_{Y \mu\sigma}$
\\
$G_{XY}\pt \leftarrow R_{X \ulgm\olgs}\pt B_{Y \mu\sigma}$
\\
$D_{X \ulgm\olgs}\pt \leftarrow H_{XY}\pt R_{Y \ulgm\olgs}\pt  + G_{XY}\pt B_{Y \ulgm\olgs}$
\\
$r^{(A)}_{ia}(\omega) \leftarrow \overline{C}_{\mu i} P_{\mu\mu'} \left[ D_{X \ulgn \olgs}\pt B_{X \nu\mu'} \right] \overline{C}_{\sigma a}$
\\
$r^{(B)}_{ia}(\omega) \leftarrow \overline{C}_{\mu i} \left[ D_{X \ulgm \olgg}\pt B_{X \gamma\sigma'} \right] Q_{\sigma'\sigma} \overline{C}_{\sigma a}$
\\
$r_{ia} += c_{os-coupling}^2 e^{\omega t\pa} \left[ - r^{(A)}_{ia}(\alpha,\omega) + r^{(B)}_{ia}(\alpha,\omega) \right]$
}
\caption{Steps for computing the doubles part of the MVP of AO-ADC(2)}
\label{AODFSOSADC2E}
\end{algorithm}

Similarly to AO-MP2, the prefactor can be significantly lowered by virtue of the incomplete Cholesky decomposition which yields the CDD-DF-SOS-ADC(2) method. Table \ref{AODFSOSADC2EOB} and \ref{AODFSOSADC2EOV} show how the intermediates in part 2E can be formed in a mixed pseudo-AO/MO basis ("OB" algorithm) or in a complete pseudo-MO basis ("OV" algorithm). While the fully transformed OV version has a lower memory footprint by reducing the dimension from $N_XN_{AO}^2$ to $N_XOV$, the overhead due to the additional Cholesky decomposition of the virtual pseudo-density might outweigh the benefit: first, the virtual space may be quite large, especially if large basis sets are used, increasing the rank of the matrix and hence the computational effort of the Cholesky decomposition. Second, the decompositions need to be recomputed for each $\omega$, and for each Laplace point. The OB version might offer a compromise between memory savings and additional overhead. 

It should be noted that the Cholesky decomposition can also be used in the Z-kernels in Table \ref{AODFSOSADC2INTERMEDIATES} to reduce its prefactor as well. 

By using local density fitting, the inherent quadratic scaling of the AO-SOS-ADC(2) method also applies for AO-DF-SOS-ADC(2) and CDD-DF-SOS-ADC(2).

%... Cholesky in K of intermediates ???

\begin{algorithm}
\setstretch{1.3}
Compute Laplace quadrature parameters $\{w\pt,t\pt\}$ for $(-\omega + \eps_i + \eps_j - \eps_a - \eps_b)^{-1}$
\\
\For{\forcond}{
Compute doubles pseudo-matrices $P\pt$ and $Q\pt$ and the Cholesky decomposition $L_{\mu \uli}$
\\
$B_{X \uli \olgs}\pt \leftarrow L\pt_{\mu\uli} B_{X \mu\nu} Q\pt_{\nu\sigma}$
\\
$v_{\uli\olgs}^{(1)(\theta)} \leftarrow L\pt_{\mu\uli} S_{\mu\nu} u_{\ulgn\olgs}$
\\
$v_{\ulgm\olgs}^{(2)(\theta)} \leftarrow u_{\ulgm\olgg} S_{\gamma\lambda} Q\pt_{\lambda\sigma}$		
\\
$R_{X \uli \olgs}\pt \leftarrow L_{\mu\uli}\pt B_{X \mu\nu} v_{\ulgn\olgs}^{(1)(\theta)} - v_{\uli\olgg}^{(2)(\theta)} B_{X \gamma\nu} Q\pt_{\nu\sigma}$
\\
$H_{XY}\pt \leftarrow L_{\mu \uli}\pt B_{X \uli\olgs}\pt B_{Y \mu\sigma}$
\\
$G_{XY}\pt \leftarrow L_{\mu \uli} R_{X \uli\olgs}\pt B_{Y \mu\sigma}$
\\
$D_{X \uli\olgs}\pt \leftarrow H_{XY}\pt R_{Y \uli\olgs}\pt  + G_{XY}\pt B_{Y \uli\olgs}$
\\
$r^{(A)}_{ia}(\omega) \leftarrow \overline{C}_{\lambda i} P_{\lambda\mu} \left[ D_{X \ulk \olgs}\pt B_{X \nu\mu} L_{\nu \ulk}\pt  \right] \overline{C}_{\sigma a}$
\\
$r^{(B)}_{ia}(\omega) \leftarrow \overline{C}_{\mu i} L_{\mu \uli} \left[ D_{X \uli \olgg}\pt B_{X\sigma\gamma}  \right] Q_{\gamma\lambda} \overline{C}_{\lambda a}$
\\
$r_{ia} += c_{os-coupling}^2 e^{\omega t\pa} \left[ - r^{(A)}_{ia}(\alpha,\omega) + r^{(B)}_{ia}(\alpha,\omega) \right]$
}
\caption{Steps for computing the doubles part of the MVP of AO-ADC(2), with Cholesky decomposition of occupied densities}
\label{AODFSOSADC2EOB}
\end{algorithm}
%
\begin{algorithm}
\setstretch{1.3}
Compute Laplace quadrature parameters $\{w\pt,t\pt\}$ for $(-\omega + \eps_i + \eps_j - \eps_a - \eps_b)^{-1}$
\\
\For{\forcond}{
Compute doubles pseudo-matrices $P\pt$ and $Q\pt$ and their Cholesky decompositions $L_{\mu \uli}\pt$ and $L_{\sigma \ola}\pt$
\\
$B_{X \uli \ola}\pt \leftarrow L\pt_{\mu\uli} B_{X \mu\nu} L\pt_{\nu\ola}$
\\
$v_{\uli\olgs}^{(1)(\theta)} \leftarrow L\pt_{\mu\uli} S_{\mu\nu} u_{\ulgn\olgs}$
\\
$v_{\ulgm\ola}^{(2)(\theta)} \leftarrow u_{\ulgm\olgg} S_{\gamma\lambda} L\pt_{\lambda\ola}$		
\\
$R_{X \uli \ola}\pt \leftarrow L_{\mu\uli}\pt B_{X \mu\nu} v_{\ulgn\olga}^{(1)(\theta)} - v_{\uli\olgg}^{(2)(\theta)} B_{X \gamma\nu} L\pt_{\nu\ola}$
\\
$H_{XY}\pt \leftarrow B_{X \uli\ola}\pt B_{Y \uli\ola}$
\\
$G_{XY}\pt \leftarrow R_{X \uli\ola}\pt B_{Y \uli\ola}$
\\
$D_{X \uli\ola}\pt \leftarrow H_{XY}\pt R_{Y \uli\ola}\pt  + G_{XY}\pt B_{Y \uli\ola}$
\\
$r^{(A)}_{ia}(\omega) \leftarrow \overline{C}_{\lambda i} P_{\lambda\mu} \left[ D_{X \ulk \ola}\pt B_{X \nu\mu} L_{\nu \ulk}  \right] L_{\sigma \ola} \overline{C}_{\sigma a}$
\\
$r^{(B)}_{ia}(\omega) \leftarrow \overline{C}_{\mu i} L_{\mu \uli} \left[ D_{X \uli \olb}\pt B_{X \gamma\sigma} L_{\sigma \olb}  \right] Q_{\gamma\lambda} \overline{C}_{\lambda a}$
\\
$r_{ia} += c_{os-coupling}^2 e^{\omega t\pa} \left[ - r^{(A)}_{ia}(\alpha,\omega) + r^{(B)}_{ia}(\alpha,\omega) \right]$
}
\caption{Steps for computing the doubles part of the MVP of AO-ADC(2), with Cholesky decomposition of occupied and virtual densities}
\label{AODFSOSADC2EOV}
\end{algorithm}

%???? DAVIDSON ???? 
